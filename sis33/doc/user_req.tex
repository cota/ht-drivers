\documentclass[12pt,a4paper]{article}
\usepackage[T1]{fontenc}
\usepackage{graphicx}
\usepackage{longtable}
\usepackage{hyperref}
\usepackage{ae,aecompl}
% The following is needed in order to make the code compatible
% with both latex/dvips and pdflatex.
\ifx\pdftexversion\undefined
\usepackage[dvips]{graphicx}
\else
\usepackage{graphicx}
\DeclareGraphicsRule{*}{mps}{*}{}
\fi

\title{SIS33xx Software Support}
\author{Emilio G. Cota $<$\href{mailto:emilio.garcia.cota@cern.ch}
  {emilio.garcia.cota@cern.ch}$>$ \\ CERN BE/CO/HT}
\date{13 August 2009}

\begin{document}

\maketitle


\section*{User Requirements}
\label{sec-1}

The purpose of this document is to make sure the work we intend to do, which
is to provide software support for the SIS33xx\footnote{SIS33xx is the name of
  a family of high-speed ADCs from Struck Innovative Systeme GmbH --
  \href{http://www.struck.de/}{http://www.struck.de/} } family, matches our
user's demands.

In the HT section we have defined a standard procedure for providing software
support for the cards under our responsibility. We intend to follow it
entirely to carry out this work.

\subsection*{Deliverables}
\label{sec-1.1}

\begin{itemize}
\item Linux/LynxOS kernel Driver
\item User-space Library to access the module. Documentation in doxygen.
\item Driver and Library test programs (not usually given to the users)
\item Code examples
\end{itemize}
An outline of the library API can be found at the end of this document. Please
spend some time on it to give us feedback.

\subsection*{SIS33xx Family}
\label{sec-1.2}

In the CO group we currently support 3 SIS33xx modules:
\href{http://wikis/display/HT/SIS3300+-+65-100+MHz+VME+FADCs}{SIS3300},
\href{http://wikis/display/HT/SIS3320+-+210+MHz+12-bit+VME+Digitizer}{SIS3320},
\href{http://wikis/display/HT/SIS3320+v250+-+250+MHz+12-bit+VME+Digitizer}
     {SIS3320-250},
whose features are outlined below. I have seen in the wiki another module,
the \href{http://wikis/display/HT/SIS3302+-+100+MHz+16-bit+VME+Digitizer}
{SIS3302},
but I haven't been able to find any spares (nor users of it) in the group.
If you need support for it, please get in touch.

\subsubsection*{SIS33xx Features}
\label{sec-1.2.1}


\begin{center}
  \begin{tabular}{lrrr}
    \hline
    &    3300  &    3320  &  3320-250  \\
    \hline
    Channels                            &       8  &       8  &         8  \\
    Bits per Sample                     &      12  &      12  &        12  \\
    Max. Samples per Channel            &  128 Ks  &   32 Ms  &     32 Ms  \\
    Internal Clock Speeds (MHz)         &     100  &     200  &       250  \\
    &      50  &     100  &       200  \\
    &      25  &      50  &       100  \\
    &    12.5  &          &        50  \\
    &    6.25  &          &            \\
    &   3.125  &          &            \\
    Min/Max External Clock Speed (MHz)  &   1/105  &  40/210  &    40/250  \\
    VME Address Space (MB)              &      16  &     128  &       128  \\
    Multi-Event                         &     Yes  &     Yes  &       Yes  \\
    \hline
  \end{tabular}
\end{center}



\emph{NB.} The SIS3320-250 is almost identical to the SIS3320; the only change
is the available sampling frequency range.

\section*{SIS33 Library API proposal}
\label{sec-2}

This will be a static library to control the three SIS modules mentioned
above. They're similar enough that we think this approach is plausible.

\subsection*{Initialisation}
\label{sec-2.1}


\subsubsection*{int sis33\_t *sis33\_open(sis33\_t *handle,
  enum sis33\_device dev, int lun)}
\label{sec-2.1.1}

Open a handle for a device of type \emph{dev}, and associate the
handle to a device given by its logical unit number (\emph{lun}). \\
return 0 on success; < 0 on failure. (e.g. lun not found.) \\
Note: \verb~sis33_t *handle~ shouldn't be dereferenced by the user.

\subsubsection*{int sis33\_close(sis33\_t *handle)}
\label{sec-2.1.2}

return 0 on success, < 0 otherwise.

\subsubsection*{int sis33\_get\_nr\_devices(enum sis33\_device dev)}
\label{sec-2.1.3}

return how many devices of type \emph{dev} are installed.

\subsubsection*{int sis33\_get\_nr\_channels(enum sis33\_device dev)}
\label{sec-2.1.4}

return the number of channels on \emph{dev} devices.

\subsubsection*{void sis33\_device\_reset(sis33\_t *handle)}
\label{sec-2.1.5}

reset the current module.

\subsection*{Error messages / Diagnostics}
\label{sec-2.2}

\subsubsection*{sis33\_errno, sis33\_strerror}
\label{sec-2.2.1}

Convert sis33lib error codes into strings.

\subsubsection*{Environment variable: SIS33LIB\_LOGLEVEL}
\label{sec-2.2.2}

Set different log levels for the library: \\
0 - print library's fatal errors \\
1 - print invalid parameters passed \\
2 - print run-time errors (e.g. ENOMEM) \\
3 - print debugging info: trace every sis33lib call \\
The debugging information is printed to stderr.

\subsubsection*{sis33\_loglevel(sis33\_t *handle, int level)}
\label{sec-2.2.3}

Set the log level to \verb~level~. \\
return previous loglevel value. \\
(See \textbf{SIS33LIB\_LOGLEVEL} above)

\subsubsection*{int sis33\_get\_version(sis33\_t *handle, char *driver,
  char *board)}
\label{sec-2.2.4}

return < 0 if there's an error; 0 otherwise.

\subsection*{Sampling Configuration}
\label{sec-2.3}

Note: per-channel configuration is very limited in these devices. If
you just are interested in a few channels, simply read data from those
and ignore the rest. \\

\subsubsection*{int sis33\_set\_trigger(sis33\_t *handle,
  enum sis33\_trigger trig)}
\label{sec-2.3.1}

Set the trigger type for all the channels on the current device. \\
The external trigger can be enabled/disabled. The Software trigger always
works.

\subsubsection*{int sis33\_set\_stop(sis33\_t *handle, enum sis33\_stop stop)}
\label{sec-2.3.2}

Set the stop type for all the channels on the current device. \\
The external stop can be enabled/disabled. Software stop always
works.

\subsubsection*{int sis33\_set\_event(sis33\_t *handle, int events,
  enum sis33\_samples samples)}
\label{sec-2.3.3}

Configure number of events and samples per event. \\
return 0 on success; < 0 if there's an error (such as wrong size). \\
Note: Each device family has a fixed set of available samples per event,
hence the \emph{enum}.

\subsubsection*{int sis33\_set\_clock(sis33\_t *handle, enum sis33\_clock clock)}
\label{sec-2.3.4}

Set the sampling clock of the device. It can be external or internal. \\
Note: the internal sampling clock range differs among SIS33xx families.
The range of allowed frequencies for the external clock also varies.

\subsubsection*{int sis33\_set\_post\_trigger(sis33\_t *handle, int postsamples,
  int post\_trig\_delay, int post\_stop\_delay)}
\label{sec-2.3.5}

Set the number of desired post samples, post-trigger delay and post-stop
delay (both in samples too). \\
The SIS33xx devices can start sampling a certain number of samples \textbf{after}
the trigger is received. The stop can also be delayed a certain number
of samples. \\
\begin{itemize}
\item Why not passing the delays in units of time? When sampling with an
  external clock, library and driver have no clue of what the sampling
  frequency is; without this frequency it's impossible to work out
  the translation from time into samples.
\end{itemize}
Note: the functions above have also mirror functions to \verb~get~ the current
values of the parameters described.

\subsection*{Acquisition}
\label{sec-2.4}


\subsubsection*{int sis33\_acq\_start(sis33\_t *handle)}
\label{sec-2.4.1}

Start the acquisition of data.

\subsubsection*{int sis33\_acq\_wait(sis33\_t *handle)}
\label{sec-2.4.2}

Start the acquisition of data and sleep until the acquisition finishes.

\subsubsection*{int sis33\_acq\_timeout(sis33\_t *handle)}
\label{sec-2.4.3}

Start the acquisition of data and sleep until the acquisition finishes
or the timeout expires. \\
Note that when the timeout has expired, the acquisition is still ongoing.
(i.e. it hasn't been cancelled.)

\subsubsection*{int sis33\_acq\_poll(sis33\_t *handle)}
\label{sec-2.4.4}

return 0 if the module is not acquiring data, < 0 otherwise.

\subsubsection*{int sis33\_force\_trigger(sis33\_t *handle)}
\label{sec-2.4.5}

Force trigger via software.

\subsubsection*{int sis33\_force\_stop(sis33\_t *handle)}
\label{sec-2.4.6}

Force stop via software.

\subsection*{Reading the Samples}
\label{sec-2.5}

\subsubsection*{int sis33\_read\_data(sis33\_t *handle, int channel,
  uint16\_t *buf, int elems)}
\label{sec-2.5.1}

Transfer the data read from \verb~channel~ to the previously-allocated
buffer \verb~buf~. \\
return number of samples transferred; < 0 on failure.

\subsection*{SIS33\{20,20-250\} only}
\label{sec-2.6}

\subsubsection*{int sis33\_set\_offset(sis33\_t *handle, int channel, int offset)}
\label{sec-2.6.1}

Set the offset of the ADC of a channel to a certain level.
return 0 on success; < 0 otherwise.
Pass channel==0 to set all the channels.
\emph{offset} is a 16-bit value which represents the desired offset to be added
by the DAC to the input stage. The default value is 0x8000, i.e. the middle
of the range. See section 4.12 of the manual for further information.

\subsubsection*{int sis33\_set\_gain(sis33\_t *handle, enum sis3320\_scale gain)}
\label{sec-2.6.2}

The gain can be set to full-scale (default) or half-scale.

\section*{Stuff I don't intend to implement}
\label{sec-3}

Please let me know if you'd like to see any of the following implemented.

\subsection*{SIS3300}
\label{sec-3.1}

\begin{itemize}
\item Gate mode: the leading edge of the trigger input initiates the trigger,
  and the trailing edge of the trigger input defines the stop. See Section
  5.2 of the SIS3300 manual.
\item Multi-event timestamps: The module has a counter to keep track of the
  time interval between stop signals received in multi-event mode.
  See Section 4.17 of the manual.
\item Averaging mode: groups of consecutive samples are averaged in the FPGAs
  before being transferred to memory. For low-speed input signals only. See
  Section 4.19.2 of the manual.
\item Multiplexer mode: used to synchronise data acquisition with slow
  external multiplexing hardware. See Section 10.1.1 of the manual.
\end{itemize}
\subsection*{Both modules}
\label{sec-3.2}

\begin{itemize}
\item Output trigger generation: an output trigger signal is generated when
  the acquired signal on a particular channel has a certain shape. (e.g.
  the signal crosses a certain threshold). Further info: \\

  \begin{itemize}
  \item SIS3300: Section 4.20 of the manual
  \item SIS3320: Section 4.27 of the manual
  \end{itemize}

\item External random clock mode: A trigger occurs after each falling
  edge of the external input clock. This falling edge is detected with
  an internal clock. The external clock doesn't need to be symmetric.

  \begin{itemize}
  \item SIS3300: Section 4.19.4 of the manual
  \item SIS3320: Section 4.5 of the manual
  \end{itemize}

\end{itemize}

\end{document}
